\subsection{Random Walk on Graphs}
In this tutorial you'll learn how to use Clairlib to perform random walk on graphs. 
To do this we'll use random\_walk.pl utility. 
This script takes a graph file in edgelist format as input and performs the random walk on 
it based on several options as described by
\\
\\
\begin{boxedverbatim}
 random_walk.pl --help
\end{boxedverbatim}
\\
\\
For example, if you have the following graph "example.graph"
\\
\\
\begin{boxedverbatim}
 1 2                
 1 4
 2 1
 2 3
 4 1
 4 3
 4 5
 5 1
 3 1
\end{boxedverbatim}
\\
\\
and the transition probabilities between the nodes are as follows "example.trans"
\\
\\
\begin{boxedverbatim}
 1 2 0.30           
 1 4 0.70
 2 1 0.65
 2 3 0.35
 4 1 0.20
 4 3 0.50
 4 5 0.30
 5 1 1.00
 3 1 1.00
\end{boxedverbatim}
\\
\\
The probabilities after walking 100 random steps for 500 rounds, can be found by running
\\
\\
\begin{boxedverbatim}
 random_walk.pl --grpah example.graph --transition_file example.trans \
 --steps 100 --rounds 500 --output example100.prob
\end{boxedverbatim}
\\
\\
The computed probabilities will be outputed to "example100.prob" and will be something like,
\\
\\
\begin{boxedverbatim}
 5 0.098            
 2 0.098         
 3 0.168         
 1 0.350       
 4 0.286
\end{boxedverbatim}
\\
\\
To compute the stationary distribution (after walking so many random steps), run the following command,
\\
\\
\begin{boxedverbatim}
 random_walk.pl --grpah example.graph --transition_file example.trans \
 --stationary --output example.prob
\end{boxedverbatim}
\\
\\
The content of the file "example.prob" after running the command will be something like this
\\
\\
\begin{boxedverbatim}
 5 0.07             
 2 0.11
 3 0.17
 1 0.37
 4 0.26
\end{boxedverbatim}