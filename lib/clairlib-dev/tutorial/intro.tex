\maketitle

\cleardoublepage

\tableofcontents

\cleardoublepage

\pagestyle{fancy}
\lhead {Clairlib}
\rhead {User Documentation}

\section{Introduction}

The University of Michigan CLAIR (Computational Linguistics and
Information Retrieval) group is happy to present version 1.08
of the Clair Library.

The Clair library is intended to simplify a number of generic tasks
in Natural Language Processing (NLP), Information Retrieval (IR), and
Network Analysis (NA).
Its architecture also allows for external software to be plugged in
with very little effort.

We are distributing the Clair library in two forms: Clairlib-core,
which has essential functionality and minimal dependence on external
software, and Clairlib-ext, which has extended functionality that
may be of interest to a smaller audience.  Depending on whether you
choose to install only Clairlib-core or both Clairlib-core and
Clairlib-ext, some of the content of this manual will not apply to
your installation.  Throughout this
document, for the sake of brevity, we will usually say ``the
Clair library'' or the more abbreviated ``Clairlib'' to refer to the
software we're distributing.

This work has been supported in part by National Institutes of Health
grants R01 LM008106 ``Representing and Acquiring Knowledge of Genome
Regulation'' and U54 DA021519 ``National Center for Integrative
Bioinformatics,'' as well as by grants IDM 0329043 ``Probabilistic and
link-based Methods for Exploiting Very Large Textual Repositories,''
DHB 0527513 ``The Dynamics of Political Representation and Political
Rhetoric,'' 0534323 ``Collaborative Research: BlogoCenter - Infrastructure
for Collecting, Mining and Accessing Blogs,'' and 0527513 ``The Dynamics of
Political Representation and Political Rhetoric,'' from the National
Science Foundation.


\subsection{Functionality}

Much can be done using Clairlib on its own.  Some of the things that
Clairlib can do are listed below, in separate lists indicating whether
that functionality comes from within a particular distribution of
Clairlib, or is made available through Clairlib interfaces, but actually
is imported from another source, such as a CPAN module, or external
software.

\subsubsection{Native to Clairlib-core}
\begin{itemize}
\item \textbf{Tokenization}:

Convert a sequence of strings into a sequence of tokens.
\item \textbf{Summarization}:

Extract content from an information and present the most important parts to the user in a condensed form.
\item \textbf{LexRank}:

Multi-document summarization, classification, and many other tasks.
\item \textbf{Biased LexRank}:

Semi-supervised passage retrieval.
\item \textbf{Document Clustering}:

Unsupervised assignment of documents into groups.
\item \textbf{Document Indexing}:

Transforming a document into an indexed form.
\item \textbf{PageRank}:

Assigning a numerical weighting to each element of a hyperlinked set of documents.
\item \textbf{Biased Pagerank}
\item \textbf{Web Graph Analysis}:

Study of link patterns emerging between documents.
\item \textbf{Network Generation}:

Generate random networks. Currently, this includes generation of Erdos-R�enyi random graphs.
\item \textbf{Power Law Distribution Analysis}
\item \textbf{Network Analysis}:
        \begin{itemize}
        \item clustering coefficient
        \item degree distribution plotting
        \item average shortest path
        \item diameter
        \item triangles
        \item shortest path matrices
        \item connected components
        \item maximum flow
        \end{itemize}
\item \textbf{Cosine Similarity}:

Similarity between two documents, represented by the vectors, by finding the cosine of the angle between them
\item \textbf{Random Walks on Graphs}
\item \textbf{Statistics}: Analyzing and generating distributions
        \begin {itemize}
        \item Distributions: Including Geometric, Gaussian, LogNormal, Zipfian and T-distribution
        \item Tests
        \end{itemize}
\item \textbf{Tf}:

Compute the Term Frequency
\item \textbf{Idf}:

Compute Inverse Document Frequency
\item \textbf{Perceptron Learning and Classification}
\item \textbf{Phrase Based Retrieval and Fuzzy OR Queries}
\item \textbf{Harmonic Functions} Computing harmonic functions using the Relaxation and Monte Carlo methods
\item \textbf{Synthetic Collection} Generation of synthetic document collections
\item \textbf{Gene Interaction Extraction} Extract interactions between genes from biomedical texts.
\end{itemize}

\subsubsection{Imported and available via Clairlib-core}
\begin{itemize}
\item Parsing
\item Stemming
\item Sentence Segmentation
\item Web Page Download
\item Web Crawling
\item XML Parsing
\item XML Tree Building
\item XML Writing
\item Statistical Parsing
\item Gene Tagging
\item N-Grams extraction
\end{itemize}

\subsection{Native to Clairlib-ext}
\begin{itemize}
\item Interfacing with Weka, a machine-learning Java toolkit
\item Latent Semantic Indexing
\item Parsing using a Charniak Parser
\item Using the Automatic Link Extractor (ALE)
\item Using Google WebSearch
\end{itemize}



\subsection{Authors}

Dragomir Radev, Mark Hodges, Anthony Fader, Mark Joseph, Joshua Gerrish, Mark Schaller, Jonathan dePeri, Bryan Gibson, Chen Huang, Amjad Abu Jbara, and Prem Ganeshkumar.

\subsection{Contributors}

Timothy Allison, Michael Dagitses, Aaron Elkiss,
Gunes Erkan, Scott Gifford, Justin Joque, Patrick Jordan,  Jung-bae Kim,
Samuela Pollack, and Adam Winkel

\subsection{Changes}

\subsubsection*{1.08 August 2009\label{1_08_August_2009}\index{1.08 August 2009}}
\begin{itemize}

\item 

Updated Clair::SynthCollection to generate synthetic documents based on (1 to 4)-grams.


\item 

Modified extract N-grams to optionally use CMU-LM.


\item 

Updated make\_synth\_collection.pl to fully utilize Clair::SyntheticCollection.


\item 

Fixed some Tokenizer issues.


\item 

Added summarize\_document.pl to the utilities.


\item 

Added summarize\_collection to the utilities.


\item 

Added learn.pl to the utilities.


\item 

Added classify.pl to the utilities.


\item 

Added extract\_features.pl


\item 

Added bigrams\_to\_rand\_doc.pl to the utilities.


\item 

Added make\_synth\_collection\_Menczer.pl to the utilities.


\item 

Added Clair::RandomWalk for random walk on graphs.


\item 

Added Clair::Harmonic for computing harmonic functions based on the Relaxation and Montecarlo methods


\item 

Added random\_walk.pl to the utilities.


\item 

Added harmonic.pl to the utilities.


\item 

Added directory\_to\_URL\_network.pl to the utilities.


\item 

Fixed a bug in the crawling code.


\item 

Added new tutorials.


\item 

Added new sections to the documentation.


\item 

Added Clair::Bio::GIN for gene interaction extraction.


\item 

Added an interface to Stanford parser in Clair::Utils::Parse


\item 

Added tag\_genes.pl to the utilities.


\item 

Added extract\_interactions.pl to the utilities.

\end{itemize}
\subsubsection*{1.07 June 2009\label{1_07_June_2009}\index{1.07 June 2009}}
\begin{itemize}

\item 

Added Clair::Network::Spectral for spectral partitioning using Fiedler Vector.


\item 

Made Clairlib independent of MEAD (MEAD is no more required for Clairlib).


\item 

Added Naive Bayes learning and classification.


\item 

Added tests for feature extraction, learning, classification.


\item 

Fixed a bug in Clair::Cluster::create\_lexical\_network().


\item 

Added sampling options to Clair::Cluster.


\item 

Added "No IDF" option and sampling capabilities to corpus\_to\_cos.pl utility.


\item 

Fixed documentation typos.


\item 

Added new tutorials to the documentation.


\item 

Fixed bug in Clair::Utils::CorpusDownload.


\item 

Added 'manual weights' option to make\_synth\_collection util.


\item 

Fixed bug in extract\_ngrams.

\end{itemize}
\subsubsection*{1.06 March 2009\label{1_06_March_2009}\index{1.06 March 2009}}
\begin{itemize}

\item 

Added Clair:Network:FordFulkerson


\item 

Added change\_perl\_path.pl  to the utilities.


\item 

Added new scripts to interface ACL Anthology Network.


\item 

Fixed a bug in split\_into\_sentences() of Clair::Document

\end{itemize}
\subsubsection*{1.05 July 2008\label{1_05_July_2008}\index{1.05 July 2008}}
\begin{itemize}

\item 

Fixed formatting bugs in CorpusDownload.pm


\item 

Added get\_predecessor\_matrix() function in Network.pm


\item 

Added get\_shortest\_path() function in Network.pm


\item 

Added  erase\_corpus.pl script


\item added erase\_isolated\_nodes.pl script
\item added --ignore-isolated-nodes in convert\_network.pl
\item added several options to print\_network\_stats.pl: a. --self-loop,
\item completed the descriptions of print\_network\_stats.pl: added the
note of --force into usage.
\item 

added sentence\_to\_docs.pl , lines\_to\_docs.pl under util folder

\end{itemize}
\subsubsection*{1.04B June 2008\label{1_04B_June_2008}\index{1.04B June 2008}}
\begin{itemize}

\item 

Added -no-duplicated-edges in convert\_network.pl


\item Added largest connected component in cos\_to\_stats.pl
\item Added full avergage shortest path in print\_network.pl
\item 

fixed divide by zero error in Network.pm, Betweeness.pm

\end{itemize}
\subsubsection*{1.04A April 2008\label{1_04A_April_2008}\index{1.04A April 2008}}
\begin{itemize}

\item 

Added Clair::Network::GirvanNewman algorithm to do hierarchical clustering


\item 

Added Clair::Network::KernighanLin algorithm to do graph partition

\end{itemize}
\subsubsection*{1.04 Feburary 2008\label{1_04_Feburary_2008}\index{1.04 Feburary 2008}}
\begin{itemize}

\item 

Added Clair::Network::AdamicAdar to compute the adamic/adar value for a given network corpus


\item 

Added Clair::ChisqIndependent to compute p-value and degree of freedom for Chi square

\end{itemize}
\subsubsection*{1.03 August 2007\label{1_03_August_2007}\index{1.03 August 2007}}
\begin{itemize}

\item Added functionality to perform community finding within weighted, undirected networks
\item Added util/chunk$\backslash$\_document.pl to break documents into smaller files by word number
\item Added option to retain punctuation for idf and tf queries
\item Added option to print out full lists of idf and tf values for a corpus
\item LexRank moved from Clair::Network to Clair::Network::Centrality::LexRank
\item LexRank use now follows the same use pattern as the other centrality modules\end{itemize}
\subsubsection*{1.02 July 2007\label{1_02_July_2007}\index{1.02 July 2007}}
\begin{itemize}

\item Distribution reorganized in standard format
\item Improved and expanded installation documentation (INSTALL)
\item Improved POD (inline) documentation
\item Additional examples
\item Updated PDF documentation\end{itemize}
\subsubsection*{1.01 May 2007\label{1_01_May_2007}\index{1.01 May 2007}}
\begin{itemize}

\item Added Phrase-based Retrieval and Fuzzy OR Queries
\item Extended Clairlib-ext with interfaces for the Cluster class and the Document class to the Weka machine learning toolkit
\item Added LSI functionality
\item Extended parsing of strings / files into Documents
\item Added perceptron learning and classification for documents\end{itemize}
\subsubsection*{1.0 RC1 April 2007\label{1_0_RC1_April_2007}\index{1.0 RC1 April 2007}}
\begin{itemize}

\item Moved all Clair modules beneath the Clair::* namespace, updated documentation
\item Improved Network Analysis, added Clustering Coefficients code
\item Added Network Generation and Statistics modules\end{itemize}
\subsubsection*{0.955 March 2007\label{0_955_March_2007}\index{0.955 March 2007}}
\begin{itemize}

\item Made it possible to distribute clairlib in two distributions, one containing core code and another containing code that may be dependent on other resources
\item Cleaned up unit tests\end{itemize}
\subsubsection*{0.953 February 2007\label{0_953_February_2007}\index{0.953 February 2007}}
\begin{itemize}

\item Fixed bugs in Clair::Cluster, Clair::Document involving stemming
\item Cleaned up t/ and test/ directories
\item Created util/ directory
\item Added scripts to util/ directory to:\begin{itemize}

\item Run a Google query and save the returned URLs to a file
\item Download files from a URL and build a corpus
\item Segment a document into sentences and build a corpus of the sentences
\item Take all documents in a directory and create a corpus
\item Index the corpus (compute TF*IDF, etc.)
\item Compute cosine similarity measures between all documents in a corpus
\item Generate networks corresponding to various cosine thresholds
\item Print network statistics about a network file
\item Generate plots of degree distribution and cosine transitions\end{itemize}

\item New methods in Clair::Network:\begin{verbatim}
    print_network_info
    get_network_info_as_string
    get_cumulative_distribution
    cumulative_power_law_exponent
    find_components
    newman_clustering_coefficient
    linear_regression
\end{verbatim}
\end{itemize}


\section{Getting Started}

\subsection{Downloading}

Clairlib can be downloaded from http://www.clairlib.org.

\subsection{Installing}



This guide explains how to install both Clairlib distributions, Clairlib-Core and Clairlib-Ext. To install Clairlib-core, follow the instructions in the section immediately below. To install Clairlib-Ext, first follow the instructions for installing
Clairlib-Core, then follow those for Clairlib-Ext itself. Clairlib-Ext requires an installed version of Clairlib-Core in order to run; it is not a stand-alone distribution.

\section{Install and Test Clairlib-Core\label{Install_and_Test_Clairlib-Core}\index{Install and Test Clairlib-Core}}
\subsection*{System Requirements\label{System_Requirements}\index{System Requirements}}


Clairlib-Core requires Perl 5.8.2 or greater. Before you proceed, confirm that the version of Perl you are running is at least this recent by entering the following at the shell prompt.

\begin{verbatim}
        perl -v
\end{verbatim}
\subsection*{Install CPAN Libraries\label{Install_CPAN_Libraries}\index{Install CPAN Libraries}}


Clairlib-Core depends on access to the following Perl modules:

\begin{description}

\item[{BerkeleyDB}] \mbox{}
\item[{Carp}] \mbox{}
\item[{File::Type}] \mbox{}
\item[{Getopt::Long}] \mbox{}
\item[{Graph::Directed}] \mbox{}
\item[{Hash::Flatten}] \mbox{}
\item[{HTML::LinkExtractor}] \mbox{}
\item[{HTML::Parse}] \mbox{}
\item[{HTML::Strip}] \mbox{}
\item[{IO::File}] \mbox{}
\item[{IO::Handle}] \mbox{}
\item[{IO::Pipe}] \mbox{}
\item[{Lingua::Stem}] \mbox{}
\item[{Lingua::EN::Sentence}] \mbox{}
\item[{Math::MatrixReal}] \mbox{}
\item[{Math::Random}] \mbox{}
\item[{MLDBM}] \mbox{}
\item[{PDL}] \mbox{}
\item[{POSIX}] \mbox{}
\item[{Scalar::Util}] \mbox{}
\item[{Statistics::ChisqIndep}] \mbox{}
\item[{Storable}] \mbox{}
\item[{Test::More}] \mbox{}
\item[{Text::Sentence}] \mbox{}
\item[{XML::Parser}] \mbox{}
\item[{XML::Simple}] \mbox{}\end{description}


There are multiple approaches to locating and installing these modules; using the automated CPAN installer, which is bundled with Perl, is perhaps the quickest and easiest. To do so, enter the following at the shell prompt:

\begin{verbatim}
        $ perl -MCPAN -e shell
\end{verbatim}


If you have not yet configured the CPAN installer, then you'll have to do so this one time. If you do not know the answer to any of the questions asked, simply hit enter, and the default options will likely suit your environment adequately. However, when asked about parameter options for the \texttt{perl Makefile.PL} command, users without root permissions or who otherwise wish to install Perl libraries within their personal \textbf{\$HOME} directory structure should enter the suggested path when prompted:

\begin{verbatim}
        Your choice:  ] PREFIX=~/perl
\end{verbatim}


This will cause the CPAN installer to install all modules it downloads and tests into \textbf{\$HOME/perl}, which means that all subdirectories of this directory that contain Perl modules will need to be added to Perl's \texttt{@INC} variable so that they will be found when needed (see next section for further explanation).



As a side note, if you ever need to reconfigure the installer, type at the shell prompt:

\begin{verbatim}
        $ perl -MCPAN -e shell
        cpan>o conf init
\end{verbatim}


After configuration (if needed), return to the CPAN shell prompt,

\begin{verbatim}
        cpan>
\end{verbatim}


and type the following to upgrade the CPAN installer to the latest version:

\begin{verbatim}
        cpan>install Bundle::CPAN
        cpan>q
\end{verbatim}


If asked whether to prepend the installation of required libraries to the queue, hit return (or enter \texttt{yes}). After quitting the shell, type the following to install or upgrade \texttt{Module::Build} and make it the preferred installer:

\begin{verbatim}
        $ perl -MCPAN -e shell
        cpan>install Module::Build
        cpan>o conf prefer_installer MB
        cpan>o conf commit
        cpan>q
\end{verbatim}


Finally, install each of the following dependencies (if you are at all unsure whether the latest versions of each have already been installed) by entering the following at the shell prompt:

\begin{verbatim}
        $ perl -MCPAN -e shell
        cpan>install BerkeleyDB
        cpan>install Carp
        cpan>install File::Type
        cpan>install Getopt::Long
        cpan>install Graph::Directed
        cpan>install HTML::LinkExtractor
        cpan>install HTML::Parse
        cpan>install HTML::Strip
        cpan>install IO::File
        cpan>install IO::Handle
        cpan>install IO::Pipe
        cpan>install Lingua::Stem
        cpan>install Math::MatrixReal
        cpan>install Math::Random
        cpan>install MLDBM
        cpan>install PDL
        cpan>install POSIX
        cpan>install Scalar::Util
        cpan>install Statistics::ChisqIndep
        cpan>install Storable
        cpan>install Test::More
        cpan>install Text::Sentence
        cpan>install XML::Parser
        cpan>install XML::Simple
\end{verbatim}
\subsection*{Configure Clairlib-Core\label{Configure_Clairlib-Core}\index{Configure Clairlib-Core}}


Download the Clairlib-Core distribution (\textbf{clairlib-core.tar.gz}) into, say, the directory \textbf{\$HOME}. Then to install Clairlib-Core in \textbf{\$HOME/clairlib-core}, enter the following at the shell prompt:

\begin{verbatim}
        $ cd $HOME
        $ gunzip clairlib-core.tar.gz
        $ tar -xf clairlib-core.tar
        $ cd clairlib-core/lib/Clair
\end{verbatim}


Then edit Config.pm, which is located in \textbf{clairlib-core/lib/Clair}. You will see the following message at the top of the file:

\begin{verbatim}
        #################################
        # For Clairlib-core users:
        # 1. Edit the value assigned to $CLAIRLIB_HOME and give it the value
        #    of the path to your installation.
        # 2. Edit the value assigned to $MEAD_HOME and give it the value
        #    that points to your installation of MEAD.
        # 3. Edit the value assigned to $EMAIL and give it an appropriate
        #    value.
\end{verbatim}


Follow those instructions. In the case of our example, we would assign

\begin{verbatim}
        $CLAIRLIB_HOME=$HOME/clairlib-core
\end{verbatim}


and

\begin{verbatim}
        $MEAD_HOME=$HOME/mead
\end{verbatim}


where \textbf{\$HOME} must be replaced by an explicit path string such as \textbf{/home/username}. Also, note that the following MEAD variables reflect the structure of a standard MEAD installation and should typically be kept as assigned:

\begin{verbatim}
        $CIDR_HOME "$MEAD_HOME/bin/addons/cidr";
        $PRMAIN    "$MEAD_HOME/bin/feature-scripts/lexrank/prmain";
\end{verbatim}
\subsection*{Test and Install the Clairlib-Core Modules\label{Test_and_Install_the_Clairlib-Core_Modules}\index{Test and Install the Clairlib-Core Modules}}


Before testing and installing the Clairlib-core modules, you'll need to modify Perl's \texttt{@INC} variable to ensure that it includes 1) paths to all Clairlib dependencies (the required libraries installed above), and 2) the path to Clairlib's own modules (in the case of our example, \textbf{\$HOME/clairlib-core/lib}). The simplest way to do this is by modifying the contents of your \texttt{PERL5LIB} environment variable from the shell prompt:

\begin{verbatim}
        $ export PERL5LIB=$HOME/clairlib-core/lib:$HOME/perl/lib     (*)
\end{verbatim}


Here \textbf{\$HOME/clairlib-core/lib} is the path to Clairlib's own modules, while \textbf{\$HOME/perl} is the path to Clairlib's required modules, installed above (assuming that path is their location). However, doing this requires that you export \texttt{PERL5LIB} each time you invoke the shell environment, so a better way to modify \texttt{@INC} is the following:

\begin{verbatim}
        $ cd $HOME
\end{verbatim}


Edit \textbf{.profile} or the appropriate configuration file for your shell environment, or create it if it does not already exist. Add \texttt{(*)} to to the file, or prepend the necessary paths using colons, as in \texttt{(*)}. Save the file and enter:

\begin{verbatim}
        $ . .profile
\end{verbatim}


This way you will not have to export \texttt{PERL5LIB} each time you invoke the
shell. Enter

\begin{verbatim}
        $ echo $PERL5LIB
\end{verbatim}


to confirm that your modifications have been applied.



Now you may test your Clairlib-Core installation. Enter its directory, in the case of our example:

\begin{verbatim}
        $ cd $HOME/clairlib-core
\end{verbatim}


Then enter the following commands to test the Clairlib-Core modules:

\begin{verbatim}
        $ perl Makefile.PL
        $ make
        $ make test
\end{verbatim}


If you would like to have the Clairlib-Core modules installed for you, and you have the necessary (root) permissions to do so, you may install them by entering the following command:

\begin{verbatim}
        $ make install
\end{verbatim}


If, on the other hand, you have only local permissions, but you have a personal perl library located at, say, \textbf{\$HOME/perl} (as described earlier), then you can install Clairlib-Core there by entering the commands:

\begin{verbatim}
        $ perl Makefile.PL PREFIX=~/perl
        $ make install
\end{verbatim}
\subsection*{Using the Clairlib-Core Modules\label{Using_the_Clairlib-Core_Modules}\index{Using the Clairlib-Core Modules}}


To use the Clairlib-Core modules in a Perl script, you must add a path to the modules to Perl's \texttt{@INC} variable. You may use either 1) \textbf{\$CLAIRLIB\_HOME/lib}, where \texttt{\$CLAIRLIB\_HOME} is the path defined in \textbf{Config.pm}, or 2) \textbf{\$INSTALL\_PATH}, where \texttt{\$INSTALL\_PATH} is a path to the location of the installed Clairlib-Core modules (if you installed them as in the previous section, immediately above). Either of these paths can be added to \texttt{@INC} either by appending the path to the \texttt{PERL5LIB} environment variable or by putting a \texttt{use lib PATH} statement at the top of the script. See the beginning of the previous section above for a detailed explanation of how to modify the \texttt{PERL5LIB} variable.

\section{Install and Test Clairlib-Ext\label{Install_and_Test_Clairlib-Ext}\index{Install and Test Clairlib-Ext}}


The Clairlib-Ext distribution contains optional extensions to Clairlib-Core as well as functionality that depends on other software. The sections below explain how to configure different functionalities of Clairlib-Ext. As each is independent of the rest, you may configure as many or as few as you wish. This section provides instructions for the installation and testing of the Clairlib-ext modules itself.

\subsection*{Sentence Segmentation using Adwait Ratnaparkhi's MxTerminator\label{Sentence_Segmentation_using_Adwait_Ratnaparkhi_s_MxTerminator}\index{Sentence Segmentation using Adwait Ratnaparkhi's MxTerminator}}


To use MxTerminator for sentence segmentation, download the installation package from:



\textsf{ftp://ftp.cis.upenn.edu/pub/adwait/jmx/jmx.tar.gz}.



Putting the tarball in, say, \textbf{\$HOME/jmx}, enter the following to unpack:

\begin{verbatim}
        $ cd $HOME/jmx
        $ gunzip jmx.tar.gz
        $ tar -xf .tar
\end{verbatim}


Uncomment and modify the following lines in \textbf{clairlib-core/lib/Clair/Config.pm}. Point \texttt{\$JMX\_HOME} to the top directory of your MxTerminator installation, and point \texttt{\$JMX\_MODEL\_PATH} to the location of your MxTerminator trained data, as for example

\begin{verbatim}
        # $JMX_HOME                "$HOME/jmx";
        # $SENTENCE_SEGMENTER_TYPE "MxTerminator";
        # $JMX_MODEL_PATH          "$HOME/jmx/eos.project";
\end{verbatim}


where \texttt{\$HOME} must be replaced by a literal path string such as \textbf{/home/username}. Note that the \textbf{/bin} directory of a Java installation must be located in your search path, or MxTerminator will not work.

\subsection*{Parsing using a Charniak Parser\label{Parsing_using_a_Charniak_Parser}\index{Parsing using a Charniak Parser}}


To use a Charniak parser with Clairlib, uncomment the following variables in \textbf{clairlib-core/lib/Clair/Config.pm} and point them to it, as for example:

\begin{verbatim}
        # Default parser and data paths for the Charniak parser for use in Parse.pm
        # (Note that CHARNIAK_DATA should end with a slash and that the other
        # paths include the executable)
        # $CHARNIAK_PATH      "/data0/tools/charniak/PARSE/parseIt";
        # $CHARNIAK_DATA_PATH "/data0/tools/charniak/DATA/EN/";
\end{verbatim}
\begin{verbatim}
        # Default path to Chunklink
        # $CHUNKLINK_PATH "/data2/tools/chunklink/chunklink.pl";
\end{verbatim}
\subsection*{Using the Weka Machine Learning Toolkit\label{Using_the_Weka_Machine_Learning_Toolkit}\index{Using the Weka Machine Learning Toolkit}}


To use the Weka Machine Learning Toolkit, a Java machine learning library, with Clairlib, download Weka from \textbf{\textsf{http://www.cs.waikato.ac.nz/ml/weka/}} and uncomment the following line in \textbf{clairlib-core/lib/Clair/Config.pm}. Point the variable to the location of Weka's \textbf{.jar} file, as for example:

\begin{verbatim}
        # $WEKA_JAR_PATH "$HOME/weka/weka-3-4-11/weka.jar"
\end{verbatim}


where \texttt{\$HOME} must be replaced by an explicit path string such as \textbf{/home/username}. Note that the \textbf{/bin} directory of a Java installation must be located in your search path, or MxTerminator will not work.

\subsection*{Using the Automatic Link Extractor (ALE)\label{Using_the_Automatic_Link_Extractor_ALE_}\index{Using the Automatic Link Extractor (ALE)}}


If you have MySQL installed and wish to use ALE, uncomment the following variables. Point \texttt{\$ALE\_PORT} at your MySQL socket, and provide the root password to your MySQL installation:

\begin{verbatim}
        # $ALE_PORT "/tmp/mysql.sock";
        # $ALE_DB_USER "root";
        # $ALE_DB_PASS "";
\end{verbatim}
\subsection*{Using Google WebSearch\label{Using_Google_WebSearch}\index{Using Google WebSearch}}


To use the Google WebSearch module, first install the CPAN module \texttt{Net::Google} (refer to the of Clairlib-Core installation instructions for further explanation) Then, uncomment the following line and provide a Google SOAP API key. Unfortunately, Google no longer gives out SOAP API keys but has moved to an AJAX Search API. If you have a SOAP API key, you can still use it, and WebSearch will still work.

\begin{verbatim}
        # $GOOGLE_DEFAULT_KEY "";
\end{verbatim}
\subsection*{Using CMU-LM tool kit.\label{Using_CMU-LM_tool_kit_}\index{Using CMU-LM tool kit.}}


The CMU-Cambridge Statistical Language Modeling toolkit is a suite of UNIX software tools to facilitate the construction and testing of statistical language models. CMU-LM is used by clairlib for N-grams extraction. It can be downloaded from \textsf{http://mi.eng.cam.ac.uk/\texttt{\~{}}prc14/toolkit.html}. Then, add the CMU-LM path to \$PATH (or modify \texttt{\~{}}/.profile):

\begin{verbatim}
        export PATH=/path/to/CMU-CAM-LM/bin:$PATH
\end{verbatim}
\subsection*{Using GENIA Tagger\label{Using_GENIA_Tagger}\index{Using GENIA Tagger}}


The GENIA tagger analyzes English sentences and outputs the base forms, part-of-speech tags, chunk tags, and named entity tags. It is used in Clair::Bio::GIN. To be able to use it, download it from \textsf{http://www-tsujii.is.s.u-tokyo.ac.jp/GENIA/tagger/} then uncomment and point the following line in Clair::Config to point to the Genia tagger home.

\begin{verbatim}
        # $GENIATAGGER_PATH = "/path/to/geniatagger";
\end{verbatim}
\subsection*{Using the Stanford Parser\label{Using_the_Stanford_Parser}\index{Using the Stanford Parser}}


To use the Stanford parser in Clairlib, download it from \textsf{http://nlp.stanford.edu/software/lex-parser.shtml} and install it as instructed in its documentation, then uncomment the following line and point it to the parser home directory.

\begin{verbatim}
        # $STANFORD_PARSER_PATH = "/path/to/stanford/parser";
\end{verbatim}
\subsection*{Configure Clairlib-Ext\label{Configure_Clairlib-Ext}\index{Configure Clairlib-Ext}}


Download the Clairlib-Ext distribution (\textbf{clairlib-ext.tar.gz}) into, for example, the directory \textbf{\$HOME}. Then to install Clairlib-Ext in \textbf{\$HOME/clairlib-ext}, enter the following at the shell prompt:

\begin{verbatim}
        $ cd $HOME
        $ gunzip clairlib-ext.tar.gz
        $ tar -xf clairlib-ext.tar
        $ cd clairlib-ext
\end{verbatim}


To test the Clairlib-Ext modules, you must first have installed the Clairlib-Core modules. Confirm that you have, then enter the following:

\begin{verbatim}
        $ perl Makefile.PL
        $ make
        $ make test
\end{verbatim}


If you would like to have the Clairlib-Ext modules installed, and you have the necessary (root) permissions to do so, you may install them by entering:

\begin{verbatim}
        $ make install
\end{verbatim}


If, on the other hand, you have only local permissions, but you have a personal perl library located at, say, \textbf{\$HOME/perl} (as described earlier), then you can install Clairlib-Ext there by entering the commands:

\begin{verbatim}
        $ perl Makefile.PL PREFIX=~/perl
        $ make install
\end{verbatim}
\subsection*{Using the Clairlib-Ext Modules\label{Using_the_Clairlib-Ext_Modules}\index{Using the Clairlib-Ext Modules}}


To use the Clairlib-Ext modules in a script, you must add a path to the modules to Perl's \texttt{@INC} variable. You may use either 1) \textbf{\$CLAIRLIB\_EXT\_HOME/lib}, where \textbf{\$CLAIRLIB\_EXT\_HOME} is the path to the top directory of your Clairlib-Ext installation, or 2) \textbf{\$INSTALL\_PATH}, where \textbf{\$INSTALL\_PATH} is a path to the location of the installed Clairlib-Ext modules (if you installed them as in the previous section). Either of these paths can be added to \texttt{@INC} either by appending the path to the \texttt{PERL5LIB} environment variable or by putting a \texttt{use lib PATH} statement at the top of the script. See the beginning of section V of the Clairlib-Core installation instructions for a detailed explanation of how to modify the \texttt{PERL5LIB} variable.

\subsection*{Change Perl Path\label{Change_Perl_Path}\index{Change Perl Path}}


To be able to run *.pl files, you may need to change the perl path in all *.pl files to the perl path on your machine. To make this easy for you, we provide a utility script "change\_perl\_path.pl" that changes the perl path in all the *.pl files in a specified directory and all its subdirectories to the path that you specify.
For example, to change the perl path in all *.pl files in the "util" directroy to /usr/local/perl/

\begin{verbatim}
        $ change_perl_path.pl util/ /usr/local/perl/
\end{verbatim}
\subsection*{Support and Documentation\label{Support_and_Documentation}\index{Support and Documentation}}


After installing Clairlib, you may access documentation for a module using the \texttt{perldoc} command, as for example:

\begin{verbatim}
        $ perldoc Clair::Document
\end{verbatim}


Each Clairlib distribution also includes a PDF tutorial. Online API documentation is available at:

\begin{verbatim}
        http://belobog.si.umich.edu/clair/clairlib/pdoc
\end{verbatim}


\section{Structure of the Clairlib Code}

The Clairlib code is divided into many modules, located in subdirectories within the \texttt{lib/Clair}
directory.

\subsection{Key Modules}
Some of the key functionality is in the \texttt{lib/Clair} directory itself:

\begin{itemize}
\item \texttt{Clair::Document} - Represents a single document
\item \texttt{Clair::Cluster} - Represents a collection of many documents
\item \texttt{Clair::Network} - Represents a network, like a graph.  The
nodes of the network may often be of type \texttt{Clair::Document}, but do
not have to be.
\item \texttt{Clair::Gen} - Works with Poisson and Power Law distributions
\item \texttt{Clair::Util} - Provides utility functions needed when using the Clair library
\item \texttt{Clair::Config} - Provides configurable constants needed by the Clair library (library paths, etc.)
\end{itemize}

Other modules in the top directory include the following:

\begin{itemize}
\item \texttt{Clair::Features} - Carry out feature selection using Chi-squared algorithm with Clair::GenericDoc
\item \texttt{Clair::Debug} - A simple class that Exports debugmsg and errmsg subs.
\item \texttt{Clair::Learn} - Implement various learning algorithms here. Default algorithm is Perceptron.
\item \texttt{Clair::Index} - Creates various indexes from supplied Clair::GenericDoc objects.
\item \texttt{Clair::Classify} - Take in the model file generated by Learn.pm and then carry out the classification.
\item \texttt{Clair::StringManip} - Majority of the string manipulation routines required by other packages.
\item \texttt{Clair::Centroid} - Compute the centroid of a cluster.
\item \texttt{Clair::Corpus} - Class for dealing with TREC corpus format data.
\item \texttt{Clair::CIDR} - Single pass document clustering.
\item \texttt{Clair::SyntheticCollection} - Generate synthetic clusters of documents.
\item \texttt{Clair::Extensions} - Versioning File for the Clairlib-ext distribution.
\item \texttt{Clair::IDF} - Handle IDF databases.
\item \texttt{Clair::SentenceFeatures} - A collection of sentence feature subroutines.
\item \texttt{Clair::RandomWalk} - Random walk on graphs.
\item \texttt{Clair::Harmonic} - Compute harmonic functions.
\end{itemize}

\subsection{Corpora Processing Modules}

Within the \texttt{lib/Clair/Utils/} directory, several modules are provided to work with corpora:

\begin{itemize}
\item \texttt{Clair::Utils::CorpusDownload} - Download corpora from a list of URLs or from a single URL as a starting point, compute IDF and TF values
\item \texttt{Clair::Utils::Idf} - Retrieve IDF values calculated by CorpusDownload
\item \texttt{Clair::Utils::Tf} - Retrieve TF values calculated by CorpusDownload
\item \texttt{Clair::Utils::TFIDFTUtils} - Provides utility functions needed for the IDF/TF calculations
\item \texttt{Clair::Utils::Robot2} - configurable web traversal engine (for web robots \& agents)
\item \texttt{Clair::Utils::LinearAlgebra}
\item \texttt{Clair::Utils::Stem} - An implementation of a stemmer
\item \texttt{Clair::Utils::MxTerminator} - Split text into sentences.
\item \texttt{Clair::Utils::ALE} - The Automatic Link Extrapolator
\end{itemize}

\subsection{Clairlib-ext Modules}
The Clairlib-ext distribution also contains the following modules in lib/Clair/Utils/:

\begin{itemize}
\item \texttt{Clair::Utils::WebSearch} - Performs Google searches and downloads files
\item \texttt{Clair::Utils::Parse} - Parse a file using the Charniak parser or use
the Chunklink tool.
\end{itemize}

\subsection{Network and Graph Processing}

Clairlib includes a large collection of network and graph processing
modules:

\begin{itemize}
\item \texttt{Clair::Network} - Network Class for the CLAIR Library
\item \texttt{Clair::NetworkWrapper} - A subclass of \texttt{Clair::Network} that wraps the C++ version of Lexrank.

\item \texttt{Clair::Network::AdamicAdar} - Adamic/Adar Algorithms, calculate the Adamic/Adar value of a network.

\item \texttt{Clair::Network::Sample} - Network sampling algorithms
\begin{itemize}
\item \texttt{Clair::Network::Sample::RandomEdge} - Random edge sampling
\item \texttt{Clair::Network::Sample::RandomNode} - Random node sampling
\item \texttt{Clair::Network::Sample::ForestFire} - Random sampling using Forest Fire model
\item \texttt{Clair::Network::Sample::SampleBase} - Abstract class for
  network sampling
\end{itemize}

\item \texttt{Clair::Network::Reader} - Different network file type readers
\begin{itemize}
\item \texttt{Clair::Network::Reader} - Abstract class for reading in network formats
\item \texttt{Clair::Network::Reader::GraphML} - Class for reading in GraphML network files
\item \texttt{Clair::Network::Reader::Pajek} - Class for reading in Pajek network files
\item \texttt{Clair::Network::Reader::Edgelist} - Class for reading in edgelist network files
\end{itemize}

\item \texttt{Clair::Network::Generator} - Random network generators
\begin{itemize}
\item \texttt{Clair::Network::Generator::GeneratorBase} - Network generator abstract class
\item \texttt{Clair::Network::Generator::ErdosRenyi} - ErdosRenyi network generator abstract class
\end{itemize}

\item \texttt{Clair::Network::Writer} - Different network file type writers
\begin{itemize}
\item \texttt{Clair::Network::Writer} - Abstract class for exporting various Network formats
\item \texttt{Clair::Network::Writer::GraphML} - Class for writing GraphML network files
\item \texttt{Clair::Network::Writer::Pajek} - Class for writing Pajek network files
\item \texttt{Clair::Network::Writer::Edgelist} - Class for writing edge list network files
\end{itemize}

\item \texttt{Clair::Network::Centrality} - Network centrality measures
\begin{itemize}
\item \texttt{Clair::Network::Centrality} - Abstract class for computing network centrality
\item \texttt{Clair::Network::Centrality::Degree} - Class for computing degree
\item \texttt{Clair::Network::Centrality::Closeness} - Class for computing closeness
centrality
\item \texttt{Clair::Network::Centrality::Betweenness} - Class for computing betweenness
centrality
\end{itemize}

\item \texttt{Clair::Network::CFNetwork} - Class for performing community finding using Newman 2004 modularity algorithm
\item \texttt{Clair::Network::KernighanLin} - Class for performing community finding and graph partition using KernighanLin algorithm
\item \texttt{Clair::Network::GirvanNewman} - Class for performing community finding using Girvan/Newman Algorithm
\item \texttt{Clair::Network::Spectral} - Class for performing spectral graph partitioning using Fiedler vector Algorithm
\item \texttt{Clair::Network::FordFulkerson} - Class for finding maximum flow using Ford/Fulkerson Algorithm
\end{itemize} % Network modules

The Network modules uses the Graph CPAN module by default, but this
other graph libraries such as Boost can be used:

\begin{itemize}
\item \texttt{Clair::GraphWrapper} - Abstract class for underlying graphs
\item \texttt{Clair::GraphWrapper::Boost} - GraphWrapper class that provides an interface to the Boost graph library
\end{itemize}

\subsection{Distributions and Statistics Modules}
There are also packages for dealing with discrete and continuous
distributions:

\begin{itemize}
\item \texttt{Clair::RandomDistribution::RandomDistributionBase} - base class for all distributions
\item \texttt{Clair::RandomDistribution::Gaussian}
\item \texttt{Clair::RandomDistribution::LogNormal}
\item \texttt{Clair::RandomDistribution::Poisson}
\item \texttt{Clair::RandomDistribution::RandomDistributionFromWeights}
\item \texttt{Clair::RandomDistribution::Zipfian}
\end{itemize}

\begin{itemize}
\item \texttt{Clair::Statistics::Distributions::TDist}
\item \texttt{Clair::Statistics::Distributions::DistBase}
\item \texttt{Clair::Statistics::Distributions::Geometric}
\end{itemize}

\subsection{ALE Modules}

\begin{itemize}
\item \texttt{Clair::ALE::Default::Tokenizer} - ALE's default tokenizer.
\item \texttt{Clair::ALE::Default::Stemmer} - ALE's default stemmer.
\item \texttt{Clair::ALE::Tokenizer}
\item \texttt{Clair::ALE::Stemmer} - Internal stemmer used by ALE

\item \texttt{Clair::ALE::Conn} - A connection between two pages, consisting of one or more links, created the the Automatic Link Extrapolator.
\item \texttt{Clair::ALE::Link} - A link between two URLs created by the Automatic Link Extrapolator.
\item \texttt{Clair::ALE::\_SQL} - Internal SQL adapter for use by ALE
\item \texttt{Clair::ALE::URL} - A URL created by the Automatic Link Extrapolator
\item \texttt{Clair::ALE::NormalizeURL}

\subsection{Political Science Modules}
 \item \texttt{Clair::Polisci} - Polisci modules
 \begin{itemize}
 \item \texttt{Clair::Polisci::AU::XMLHandler}
 \item \texttt{Clair::Polisci::US::XMLHandler}
 \item \texttt{Clair::Polisci::US::Connection} - Read records from the US polisci database
 \item \texttt{Clair::Polisci::Speaker} - An object representing a hansard speaker
 \item \texttt{Clair::Polisci::Record} - An object representing a hansard record

 \item \texttt{Clair::Polisci::Graf} - An object representing a hansard graf
 \item \texttt{Clair::Polisci::AustralianParser} - A class for parsing Australian hansard html.
 \end{itemize}

\subsection{Mead Interfacing Modules}

\item \texttt{Clair::MEAD::DocsentConverter} - Document => Mead Cluster converter
\item \texttt{Clair::MEAD::Summary} - Access to a MEAD summary
\item \texttt{Clair::MEAD::Wrapper} - A perl module wrapper for MEAD

\subsection{Bio Modules}
\begin{itemize}
\item \texttt{Clair::Bio} - Bio utilities
\item \texttt{Clair::Bio::EUtils::ESearchHandler} - An XML handler for parsing ESearch results
\item \texttt{Clair::Bio::EUtils::ESearch} - A Perl interface to the ESearch utility
\item \texttt{Clair::Bio::EUtils} - A base class for Bio::EUtils objects
\item \texttt{Clair::Bio::Connection} - Connect to the Bio database using SOAP
\item \texttt{Clair::Bio::GeneRIF} - Perl module for parsing GeneRIF files
\item \texttt{Clair::Bio::GIN} - Gene interaction extraction.
\item \texttt{Clair::Bio::GIN::Data} - Interface to data files used for interaction extraction.
\item \texttt{Clair::Bio::GIN::Interaction} - Data structure for representing a gene interaction.
\end{itemize}

\subsection{Information Retrieval Modules}

\item \texttt{Clair::IR} - Basic Information Retrieval operations

\subsection{LinkPolicy Modules}

\item \texttt{Clair::LinkPolicy} - Different document linking policies
\begin{itemize}
\item \texttt{Clair::LinkPolicy::MenczerMacro} - Class implementing the Menczer Micro link model
\item \texttt{Clair::LinkPolicy::LinkPolicyBase} - Base class for creating corpora from collections
\item \texttt{Clair::LinkPolicy::RadevPAMixed} - Class implementing the RadevPAMixed  link model
\item \texttt{Clair::LinkPolicy::MenczerPAMixed} - Class implementing the MenczerPAMixed Micro link model
\item \texttt{Clair::LinkPolicy::RadevMicro} - Class implementing the Radev Micro link model
\item \texttt{Clair::LinkPolicy::BarabasiAlbert} - Class implementing the Barabasi Albert link model.
\item \texttt{Clair::LinkPolicy::WattsStrogatz} - Class implementing the Watts/Strogatz link model
\item \texttt{Clair::LinkPolicy::ErdosRenyi} - Class implementing the Erdos Renyi link model
\end{itemize}

\subsection{Sentence Segmentation Modules}

\item \texttt{Clair::SentenceSegmenter} - Sentence segmentation
\begin{itemize}
\item \texttt{Clair::SentenceSegmenter::SentenceSegmenter}
\item \texttt{Clair::SentenceSegmenter::Text}
\end{itemize}

\subsection{Generic Document Modules}
\item \texttt{Clair::GenericDoc} - Generic document representations and parsing modules
\begin{itemize}
\item \texttt{Clair::GenericDoc} - A class to standardize and create generic representation of documents.
\item \texttt{Clair::GenericDoc::html} - A submodule that strips out html tags.
\item \texttt{Clair::GenericDoc::shakespeare} - specialized to parse shakespeare html files.
\item \texttt{Clair::GenericDoc::octet\_stream} - A submodule that parses xml and converts it into a hash
\item \texttt{Clair::GenericDoc::sports} - A specialized module for parsing docs for hw2
\item \texttt{Clair::GenericDoc::xml} - A submodule that parses xml and converts it into a hash
\item \texttt{Clair::GenericDoc::plain} - A submodule that returns the document as is.
\end{itemize}

\subsection{Other Modules}

\item \texttt{Clair::CIDR::Wrapper} - A wrapper script for the original cidr script
\item \texttt{Clair::Nutch::Search} - A class for performing simple Nutch searches.


\item \texttt{Clair::Interface::Weka} - Interfacing with Weka, a machine-learning Java toolkit.
\item \texttt{Clair::Index::mldbm} - A submodule that gets dynamically loaded by Index.pm.
\item \texttt{Clair::Index::dirfiles} - Builds the index into the filesystem namespace.


\item \texttt{Clair::Algorithm::LSI} - Latent Semantic Indexing.
\item \texttt{Clair::Info::Query} - A module that implements different types of queries.
\item \texttt{Clair::Info::Stats}

\end{itemize}

Many of the above modules are described in more details in the following section. 