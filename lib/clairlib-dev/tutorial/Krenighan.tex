\subsection{Graph Partitioning Using Krenighan-Lin Algorithm}

In this tutorial, you will use Clair::Network::KrenighanLin to partition the Karate Club network into two partitions. For more information about Karate Club network, see \textbf{Spectral Partitioning Using Fiedler Vector tutorial}.

\subsubsection{Read in the network file and create a Clair::Network object}

First you need to create a \emph{Clair::Network::Reader::GML} object
\\
\\
\begin{boxedverbatim}
 use Clair::Network::Reader::GML;
 my $reader=new Clair::Network::Reader::GML();
\end{boxedverbatim}
\\
\\
Then, pass the network filename to the \emph{read\_network} subroutine via the \emph{\$reader object}. This will return a \emph{Clair::Network} object.
\\
\\
\begin{boxedverbatim}
 use Clair::Network;
 my $filename = "karate.gml";
 my $net = $reader->read_network($filename);
\end{boxedverbatim}
\\
\subsubsection{Create a Clair::Network::KrenighanLin object}

Create a new \emph{Clair::Network::KrenighanLin} by calling its constructor
\\
\\
\begin{boxedverbatim}
 use Clair::Network::KrenighanLin;
 $KL = new Clair::Network::KrenighanLin($net);
\end{boxedverbatim}
\\
\\
\subsubsection{Partition the graph}

To partition the graph, simply call \emph{generatePartition()} subroutine via the \emph{\$GN} object. \emph{generatePartition()} returns the result as a hash.
\\
\\
\begin{boxedverbatim}
 my $graphPartition = $KL->generatePartition();
\end{boxedverbatim}
\\
\\
\emph{\$graphPartition} is a hash with "node id" as key and "partition number (0/1)" as value.

You can use the dumper to print the contents of \emph{\$graphPartition}.
\\
\\
\begin{boxedverbatim}
 use Data::Dumper
 print Dumper($graphPartition);
\end{boxedverbatim}
\\
\\
The output will be something like
\\
\\
\begin{boxedverbatim}
 $VAR1 = {
          '33' => 0,
          '32' => 0,
            .
            .
          '19' => 0,
          '5' => 1
        };
\end{boxedverbatim} 