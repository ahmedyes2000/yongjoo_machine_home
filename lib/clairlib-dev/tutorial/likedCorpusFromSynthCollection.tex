\subsection{Create a Linked Corpus Out of a Pre-generated Synthetic Collection}

In this tutorial, we will create a linked corpus out of a pre-generated synthetic collection using the utility \emph{link\_synthetic\_collection.pl}. First off, we need to generate a synthetic collection to pass to this utility. We will use \emph{make\_synth\_collection.pl} to create a collection of documents called \emph{SynthCollection} in the directory \emph{synth\_out}, as demonstrated earlier in this tutorial:
\\
\\
\begin{boxedverbatim}
mkdir source
cd source
wget -r -nd -nc http://belobog.si.umich.edu/clair/corpora/chemical
cd ..
directory_to_corpus.pl --corpus chemical --base produced \
--directory source
index_corpus.pl --corpus chemical --base produced
make_synth_collection.pl --output SynthCollection --directory synth_out \
--corpus chemical --base produced --size 20 --term-policy zipfian \
--term-alpha 1 --doclen-policy mirror --verbose
\end{boxedverbatim}
\\
\\
Now, we can use \emph{link\_synthetic\_collection.pl} to link this collection of documents and create a corpus. \emph{link\_synthetic\_collection.pl} provides a number of policies to use when generating the synthetic corpus. Each link policy requires various arguments, as explained by the command:
\\
\\
\begin{boxedverbatim}
link_synthetic_collection.pl --help
\end{boxedverbatim}
\\
\\
For this tutorial, we will use the Watts-Strogatz option, which requires the arguments \emph{-p} (link probability) and \emph{-k} (number of neighbors per node).
\\
\\
\begin{boxedverbatim}
link_synthetic_collection.pl -n SynthCorpus -b synth_corpus\
 -c SynthCollection -d synth_out -l watts -p 0.42 -k 3
\end{boxedverbatim}
\\
\\
This command will look in the directory \emph{synth\_out/} for a collection of documents called \emph{SynthCollection}. Then, it will create a directory called \emph{synth\_corpus/} and link \emph{SynthCollection} to create a corpus called \emph{SynthCorpus}. 