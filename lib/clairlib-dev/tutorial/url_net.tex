\subsection{Generate a URL-Based Network out of a Hyperlinked Dataset}
In this tutorial we will use some Clairlib utils to generate a URL-based network out of a hyperlinked dataset without the need to index it since indexing takes a long time in large datasets. We will show two ways to do this:

\subsubsection{By creating a CL corpus first}

This way involves three steps:

\begin{itemize}
  \item  Create a corpus out of the dataset
  \item  Generate the links database of the corpus
  \item  Create the URL-based network
\end{itemize}

\textbf{Create a Corpus out of the dataset}

For the purpose of this tutorial we'll build a corpus by downloading the dataset files from internet. We will use the chemical elements dataset as an example.
\\
\\
\begin{boxedverbatim}
 download_urls.pl -c chemical \
 -i http://belobog.si.umich.edu/clair/corpora/chemical \
 -b produced
\end{boxedverbatim}
\\
\\
If you have the files already downloaded and stored in some directory on your machine, you can use
\\
\\
\begin{boxedverbatim}
 directory_to_corpus.pl --corpus chemical \
 --directory source --base produced --type html
\end{boxedverbatim}
\\
\\
where "source" the directory where the dataset is located.

\textbf{Generate the links database of the corpus}

To do this, we'll use index\_corpus.pl utility. However, we'll pass some parameters that instruct it to skip the indexing part and only build the files needed for the next step (i.e. building the URL network of the corpus)
\\
\\
\begin{boxedverbatim}
 index_corpus.pl --corpus checmial \
 --base produced --notf --noidf --nostats
\end{boxedverbatim}
\\
\\
The --notf, the --noidf, and the --nostats arguments ask the code to skip the indexing steps.


\textbf{Create the URL-Based Network}

To create the URL network, we'll use the corpus\_to\_network.pl util as follows
\\
\\
\begin{boxedverbatim}
 corpus_to_network.pl -c chemical -b produced -o chemical.graph
\end{boxedverbatim}
\\
\\
Where chemical.graph is the name of the resulting graph file.


\subsubsection{Without creating a corpus}

To do this, lets download the chemical files and store them into a directory "chemical\_src"
\\
\\
\begin{boxedverbatim}
 mkdir chemical_src
 cd chemical_src
 wget -r -nd -nc http://belobog.si.umich.edu/clair/corpora/chemical
 cd ..
\end{boxedverbatim}
\\
\\
Then, we will use the directory\_to\_URL\_network.pl utility as follows:
\\
\\
\begin{boxedverbatim}
 directory\_to\_URL_network.pl --directory chemical_src \
 --output chemical.graph
\end{boxedverbatim}