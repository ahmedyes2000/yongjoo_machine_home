\section{Related Work}

1) Relation to Zhu's method harmonic function and Gaussian fields

-- distribution assumption

Zhu's method assumes Gaussian random fields, the propability distribution of random
walk is a continuous Gaussian distribution on the reverse of the distance
between any two nodes. The graph is fully connected. 


It is a good model when the geometric distance is well defined.

In the bipartite graph model, the propability of one random step is propotional to the number of
common features shared by the two example nodes. It is actually the dot product
of two examples (recall that an example is a vector of dimention $m$).
If the number of features connected to every example 
is same, i.e. every example has same magnitude, like in the pp attachment case,
then it's also cosine). The random walk propability is a discrete distribution.

-- advantage of tumbl

graph construction is cheap: $O(nm)$, Zhu's method needs
to compute a $n \times n$ weight matrix, where each entry contains calculation
of distance, so the complexity is $O(n^2m)$

2) random walk/ label propagation applied to NLP tasks

Todo: check the related work cited in the word polarity paper, 
 e.g. Rao (2009) for sentiment classification


