\section{Conclusion and Future Work}
\label{sec:con}
In this paper we present a framework based on the HITS algorithm that employs heterogeneous information (i.e., citations and source texts) to generate surveys of scientific paradigms. Using both Rouge and nugget-based evaluations, we show that our proposed system, Surveyor, generates summaries that have higher quality than the state-of-the-art methods when compared with end of chapter summaries and historical notes in Jurafsky and Martin NLP textbook.

In our work, we have used Jurafksy and Martin's end of chapter summaries as the gold standard written by written experts. We believe that the area of text summarization, and especially summarizing scholarly work can benefit from a wide range of expert written summaries that are produced more naturally, outside the context of multi-document summarization experiments. Other examples of such a gold standard source include ``further reading'' sections in the leading Information Retrieval textbook~\cite{manning07}, or survey papers published occasionally in journals such as Computational Linguistics.

One of the authors of this paper organized an NLP seminar previously. As part of the seminar, the students in the class took turns to present surveys of specific topics in NLP and Information Retrieval (IR) and wrote chapter-length surveys of their topics. 
In future work, we plan to make use of the surveys written by NLP students as  gold standard in evaluations. Compared to the chapters from JM book, these topics are more  specific and close to the latest development in NLP and IR. Examples include  Sentiment and Polarity Extraction, Science Maps, Spectral graph-based methods for NLP,  Information Diffusion In Graphs, Financial Networks and Query Expansion. 

In current work, we are using the papers cited in each chapter of the JM textbook
as seed source papers (i.e. we assume that the set of seminal papers on each topic are known).  However in the science community, there are  thousands more papers that are related to a given topic.  In the future,  we will work on a method of automatically identifying the most influential papers that represent a specific topic from the vast range of publications.
